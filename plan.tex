\section{Project plan}

Prosjektplan. Må inkludere ressursbehov.

Gaps-analyse: Identifisere gap mellom en gitt anbefaling/rammeverk og dagens situasjon. Fra dette følger et sett med målsetninger som inkluderes i de generelle målsetningene for emnet. Deretter identifiseres delprosjekter som skal identifisere hvordan man realiserer disse målsetningene i studieprogrammet.

- checkpoint: skal prosjektplanen revideres.

Gapsanalysen identfiserer endringer som bør gjennomføres og prosjekter som må kjøres.

- Forankring x 2 (fagmiljø, ledelse)

Prosjekter:

Utarbeide forslag til programstruktur - tilbakemelding.

Forslag til dokumentstruktur.

Prosjekt: behov i fagmiljø fra fellesemner (og andre ferdigheter)

Beregningsorientering

Læringsmiljø/pedagogikk

Eksprimentelle ferdigheter.

Resultat: Hoveddokument, Realisering, Engasjert stab.

