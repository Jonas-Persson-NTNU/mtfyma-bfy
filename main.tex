\documentclass[a4paper, oneside, 12pt]{memoir}
% Use twoside for a version compatible with two-sided printing.

\input{preamble}

\begin{document}

\frontmatter
\tableofcontents

\part{Innledning}
% Rollen og strukturen til dette dokumentet.

\mainmatter



\part{Målsetning}

Denne delen beskriver målsetningene med studieprogrammene MTFYMA og BFY. Målsetningen beskrives på ulike nivå, fra et overordnet nivå til et detaljnivå som vil være en del av et emne.

% Overordnet målsetning.
\include{section/program-descriptions}
% Beskrivelse av egenarten til programmene.

\chapter{Detaljerte målsetninger}
Dette kapittelet beskriver kompetansen som studentene skal opparbeide seg på detaljert nivå.

%Kompetanseområder
% Fysikk og matematikk
% Analytiske ferdigheter
% Modellering
% Beregning
% Eksperimentelt
% Kommunikasjon
% Kritisk innsamling av informasjon
% Samarbeid
% Planlegging
% Læring
% Andre fagområder
%objectives from EPS recommendation for a Bachelor programme in physics
Classical mechanics
\begin{itemize}
    \item Newton’s laws and conservation laws including rotation
    \item Newtonian gravitation to the level of Kepler’s laws
\end{itemize}
Thermodynamics and kinetic theory of gases
Zeroth,first and second laws of thermodynamics to include:
\begin{itemize}
    \item Temperature scales,work,internal energy and heat capacity
    \item Entropy,free energies and the Carnot cycle
    \item Kinetic theory of gases and the gas laws to the level of the van der Waals equation
    \item The Maxwell-Boltzmann distribution
    \item Statistical basis of entropy
    \item Changes of state
\end{itemize}
Special relativity
to the level of
\begin{itemize}
    \item Lorentz transformations and the energy-momentum relationship
\end{itemize}
Advanced classical mechanics
\begin{itemize}
    \item Basic Lagrangian and Hamiltonian mechanics
\end{itemize}
Background to quantum mechanics
\begin{itemize}
    \item Blackbody radiation
    \item Photoelectric effect
    \item Wave-particle duality
    \item Heisenberg’s Uncertainty Principle
\end{itemize}
Oscillations & waves
\begin{itemize}
    \item Free,damped,forced and coupled oscillations to include resonance and normal modes
    \item Waves in linear media to the level of group velocity
    \item Waves on strings soundwaves and electro-magnetic waves
    \item Doppler effect
\end{itemize}
Basic optics
\begin{itemize}
    \item Geometrical optics to the level of simple optical systems
    \item The electromagnetic spectrum
    \item Interference and diffraction at single and multiple apertures
    \item Dispersion by prisms and diffraction gratings
    \item Optical cavities and laseraction
    
\end{itemize}
Electromagnetism
\begin{itemize}
    \item Electrostatics and magnetostatics
    \item DC and AC circuit analysis to the level of complex impedance,transients and resonance
     \item Gauss,Faraday,Ampère,Lenz and Lorentz laws to the level of their vector expression

\end{itemize}
Advanced Electrodynamics and Optics
\begin{itemize}
    \item Maxwell’s equations and plane electromag-netic waves;Poynting vector
    \item Polarisation of waves and behaviour at plane interfaces
    \item 
    \item 
    \item 
\end{itemize}
Quantum mechanics
Schrödinger wave equation to include:
\begin{itemize}
    \item Wave function and its interpretation
    \item Standard solutions and quantum numbers to the level of the hydrogen atom
    \item Tunnelling
    \item First order time independent perturbation theory
\end{itemize}
Statistical mechanics
\begin{itemize}
    \item Bose-Einstein and Fermi-Dirac distributions
    \item Density of states and partition function
    \item 
    \item 
\end{itemize}
Atomic,nuclear and particlephysics
\begin{itemize}
    \item Quantum structure and spectra of simple atoms
    \item Nuclear masses and bindingenergies
    \item Radioactive decay,fission and fusion
    \item Pauli exclusion principle,fermions and bosons and elementary particles
    \item Fundamental forces and the Standard Model
\end{itemize}
Solid state physics
\begin{itemize}
    \item Mechanical properties of matter to include elasticity and thermal expansion
    \item Inter-atomic forces and bonding
    \item Phonons and heat capacity
    \item Crystal structure and Bragg scattering
    \item Electron theory of solids to the level of simple band structure
    \item Semiconductors and doping 
    \item Magnetic properties of matter
\end{itemize}
Laboratory work
\begin{itemize}
    \item plan ane xperimental investigation
    \item use apparatus to acquire experimental data;
    \item analyse data using appropriate techniques;
    \item determine and interpret the measurement uncertainties (both systematic and random) in ameasurement or observation;
    \item report the results of an investigation and
    \item Understand how regulatory issues such as health and safety influence scientific experimentation and observation.
\end{itemize}
Project work
The objectives of such project work will include most of the following:
\begin{itemize}
    \item investigation of a physics-based or physics-related problem
    \item planning, management and operation of an investigation to test a hypothesis
    \item development of information retrieval skills
    \item carrying out a health and safety assessment
    \item establishment of co-operative working prac-tices with colleagues
    \item design, assembly and testing of equipment or software generation and informed analysis of data and a critical assessment of experimental(or other) uncertainties
\end{itemize}


\part{Gjennomføring}

Denne delen beskriver \emph{hvordan} man skal oppnå de målsetninger man har satt seg i forrige del av dokumentet. Denne delen starter med å beskrive det læringsmiljøet man ønsker å etablere og dernest hvilke undervisnings- og vurderingsformer som bør brukes for å bygge dette læringsmiljøet.

	\chapter{Læringsmiljø}
	
	\chapter{Undervisningsformer}
	
	\chapter{Vurderingsformer}

\part{Oppbygging}

Denne delen beskriver hvordan man hensiktsmessig kan bygge opp studiet for å oppnå de målsetninger som ble satt i første del, gjennom de læringsaktivitetene som ble beskrevet i andre del. Beskrivelsen starter med et overordnet nivå (emnevegg) og går til enkeltelementer i emnene.

\appendix

\renewcommand{\appendixtocname}{Tillegg}
\renewcommand{\appendixpagename}{Tillegg}
\part*{Tillegg}
\addcontentsline{toc}{part}{Tillegg}
%\appendixpage*

\chapter{Utviklingsplan}

Denne delen inneholder konkrete planer for endringsprosjekter med mål om realisere målsetningene med studieprogrammet.

\section{Prosjektplan}

	\input{appendix/plan/plan}

\section{Prosjekter}
	
	\input{appendix/plan/project-gapanalysis2022}
	\input{appendix/plan/project-vision2022}

% Langtidsplan

% Forankring

% Hvilket behov ser faggruppene for kunnskap og ferdigheter som er nødvendige for å kunne ta fag som faggruppen er 'ansvarlig' for i høyere årskurs. Hvilke ferdigheter er nødvendige for å ta masteroppgaver? Er det tidligere observert noen generelle mangler?

% Hvordan bør dette dokumentet struktureres?

\section{Ressursbehov}

% Oppsummering av ressursbehov

\chapter{Gapsanalyser}

Dette kapittelet inneholder en sammenlikning av studieprogrammene i forhold til ulike rapporter som har analysert hva som vil være ettertraktet kompetanse hos fremtidige studenter. I tillegg er det gjort et forsøk på et sammendrag av tiltak basert på gapsanalysen i Forslag til endringer.

\input{appendix/gaps/changes}

\section{Fremtidens teknologistudium}
	\input{appendix/gaps/fts-principles}
	\input{appendix/gaps/fts-competencies}

%Dybdeevaluering 2020
\input{appendix/gaps/evaluation2020-summary}

% \section{CDIO}
% Gap analysis relative to CDIO standards

% \section {IOP, APS, ...}
% Gap analysis relative to professional societies

% \chapter{Organisasjon}
% Describe the organization that is necessary to maintain this document and ensure its continued use.

% Hvordan sørge for at prosjektet ikke er personavhengig.

% \chapter{Dokumenthåndtering}
% Description of how the document is manged, updated, communicated.
% Plasseres på github fordi ...., github bruker til ... overføres når...

\end{document}
