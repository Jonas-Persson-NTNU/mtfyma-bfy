\documentclass[a4paper, oneside, 12pt]{memoir}
% Use twoside for a version compatible with two-sided printing.

% Layout
\usepackage[margin=2cm]{geometry} 

% Language
\usepackage[norsk]{babel}

% References
\usepackage{hyperref}

% For block commenting
\usepackage{comment}

% Typesetting
\usepackage{csquotes}


\begin{document}

\frontmatter
\tableofcontents

\part{Innledning}
% Rollen og strukturen til dette dokumentet.

\mainmatter



\part{Målsetning}

Denne delen beskriver målsetningene med studieprogrammene MTFYMA og BFY. Målsetningen beskrives på ulike nivå, fra et overordnet nivå til et detaljnivå som vil være en del av et emne.

% Overordnet målsetning.
\section{Overordnet målsetning for programmene}

Felles for MTFYMA og BFY er at de skal gi en solid innføring i matematikk of fysikk og de nødvendige ferdigheter som trengs for å kunne utøve disse fagfeltene.

MTFYMA gir i tillegg til det faglige fundamentet også kunnskap og ferdigheter ikke-faglige ferdigheter som er viktige som ansatt i en bedrift. Studiet består i stor grad obligatoriske for effektivt å kunne oppnå faglig bredde.

BFY tilstrekkelig valgfrihet mot slutten av studiet til å kunne begynne spesialisering mot ulike fagfelt både innen og utenfor fysikk, og sørge for at studenten kan kvalifisere seg til opptak mot disse studiene. Kombinasjonen BFY + MSPHYS gir mulighet for dypere spesialisering i fagområder innen fysikk, mens muligheten for kombinasjonen BFY + annet masterprogram, gir en bredde av kompetanseprofiler.


% Beskrivelse av egenarten til programmene.

\chapter{Detaljerte målsetninger}
Dette kapittelet beskriver kompetansen som studentene skal opparbeide seg på detaljert nivå.

%Kompetanseområder
% Fysikk og matematikk
% Analytiske ferdigheter
% Modellering
% Beregning
% Eksperimentelt
% Kommunikasjon
% Kritisk innsamling av informasjon
% Samarbeid
% Planlegging
% Læring
% Andre fagområder
%objectives from EPS recommendation for a Bachelor programme in physics
Classical mechanics
\begin{itemize}
    \item Newton’s laws and conservation laws including rotation
    \item Newtonian gravitation to the level of Kepler’s laws
\end{itemize}
Thermodynamics and kinetic theory of gases
Zeroth,first and second laws of thermodynamics to include:
\begin{itemize}
    \item Temperature scales,work,internal energy and heat capacity
    \item Entropy,free energies and the Carnot cycle
    \item Kinetic theory of gases and the gas laws to the level of the van der Waals equation
    \item The Maxwell-Boltzmann distribution
    \item Statistical basis of entropy
    \item Changes of state
\end{itemize}
Special relativity
to the level of
\begin{itemize}
    \item Lorentz transformations and the energy-momentum relationship
\end{itemize}
Advanced classical mechanics
\begin{itemize}
    \item Basic Lagrangian and Hamiltonian mechanics
\end{itemize}
Background to quantum mechanics
\begin{itemize}
    \item Blackbody radiation
    \item Photoelectric effect
    \item Wave-particle duality
    \item Heisenberg’s Uncertainty Principle
\end{itemize}
Oscillations & waves
\begin{itemize}
    \item Free,damped,forced and coupled oscillations to include resonance and normal modes
    \item Waves in linear media to the level of group velocity
    \item Waves on strings soundwaves and electro-magnetic waves
    \item Doppler effect
\end{itemize}
Basic optics
\begin{itemize}
    \item Geometrical optics to the level of simple optical systems
    \item The electromagnetic spectrum
    \item Interference and diffraction at single and multiple apertures
    \item Dispersion by prisms and diffraction gratings
    \item Optical cavities and laseraction
    
\end{itemize}
Electromagnetism
\begin{itemize}
    \item Electrostatics and magnetostatics
    \item DC and AC circuit analysis to the level of complex impedance,transients and resonance
     \item Gauss,Faraday,Ampère,Lenz and Lorentz laws to the level of their vector expression

\end{itemize}
Advanced Electrodynamics and Optics
\begin{itemize}
    \item Maxwell’s equations and plane electromag-netic waves;Poynting vector
    \item Polarisation of waves and behaviour at plane interfaces
    \item 
    \item 
    \item 
\end{itemize}
Quantum mechanics
Schrödinger wave equation to include:
\begin{itemize}
    \item Wave function and its interpretation
    \item Standard solutions and quantum numbers to the level of the hydrogen atom
    \item Tunnelling
    \item First order time independent perturbation theory
\end{itemize}
Statistical mechanics
\begin{itemize}
    \item Bose-Einstein and Fermi-Dirac distributions
    \item Density of states and partition function
    \item 
    \item 
\end{itemize}
Atomic,nuclear and particlephysics
\begin{itemize}
    \item Quantum structure and spectra of simple atoms
    \item Nuclear masses and bindingenergies
    \item Radioactive decay,fission and fusion
    \item Pauli exclusion principle,fermions and bosons and elementary particles
    \item Fundamental forces and the Standard Model
\end{itemize}
Solid state physics
\begin{itemize}
    \item Mechanical properties of matter to include elasticity and thermal expansion
    \item Inter-atomic forces and bonding
    \item Phonons and heat capacity
    \item Crystal structure and Bragg scattering
    \item Electron theory of solids to the level of simple band structure
    \item Semiconductors and doping 
    \item Magnetic properties of matter
\end{itemize}
Laboratory work
\begin{itemize}
    \item plan ane xperimental investigation
    \item use apparatus to acquire experimental data;
    \item analyse data using appropriate techniques;
    \item determine and interpret the measurement uncertainties (both systematic and random) in ameasurement or observation;
    \item report the results of an investigation and
    \item Understand how regulatory issues such as health and safety influence scientific experimentation and observation.
\end{itemize}
Project work
The objectives of such project work will include most of the following:
\begin{itemize}
    \item investigation of a physics-based or physics-related problem
    \item planning, management and operation of an investigation to test a hypothesis
    \item development of information retrieval skills
    \item carrying out a health and safety assessment
    \item establishment of co-operative working prac-tices with colleagues
    \item design, assembly and testing of equipment or software generation and informed analysis of data and a critical assessment of experimental(or other) uncertainties
\end{itemize}


\part{Gjennomføring}

Denne delen beskriver \emph{hvordan} man skal oppnå de målsetninger man har satt seg i forrige del av dokumentet. Denne delen starter med å beskrive det læringsmiljøet man ønsker å etablere og dernest hvilke undervisnings- og vurderingsformer som bør brukes for å bygge dette læringsmiljøet.

	\chapter{Læringsmiljø}
	
	\chapter{Undervisningsformer}
	
	\chapter{Vurderingsformer}

\part{Oppbygging}

Denne delen beskriver hvordan man hensiktsmessig kan bygge opp studiet for å oppnå de målsetninger som ble satt i første del, gjennom de læringsaktivitetene som ble beskrevet i andre del. Beskrivelsen starter med et overordnet nivå (emnevegg) og går til enkeltelementer i emnene.

\appendix

\renewcommand{\appendixtocname}{Tillegg}
\renewcommand{\appendixpagename}{Tillegg}
\part*{Tillegg}
\addcontentsline{toc}{part}{Tillegg}
%\appendixpage*

\chapter{Utviklingsplan}

Denne delen inneholder konkrete planer for endringsprosjekter med mål om realisere målsetningene med studieprogrammet.

\section{Prosjektplan}

	Her følger en prosjektplan for revisjon av programmene.

2022 

06-07 Utvikle gapsanalyser relativt til rapporter/anbefalinger: FTS (prinsipper og kompetanseprofiler), dybdeevaluering 2020. Dette skal resultere i en liste med konkrete anbefalinger om endring av innhold i studieprogrammet.  

07 Presentasjon for fagmiljøene

Høst Innhold- og strukturprosjekter: Overordnet struktur, beregning, eksperimentelt, læringsmiljø, vurderingsformer, undervisningsformer, mekanikk, elmag, matematikk, ønsker faggrupper.

Revider prosjektplan. Større/mindre aktiviteter

Ressursbehov er 20\% frikjøp i prosjekgruppen (50\% i tillegg til programledere) De andre komiteene leverer arbeider som ikke er for arbeidskrevende. 

2023 

Vår Overordnet struktur legges litt fastere, undervisningsformer.

Ressursbehov. som over. Det er relativt usikkert hva ressursbehovet her er, ettersom dette er en relativt ny arbeidsmetode.

% Det kan være behov her for en gruppe som kan gå litt dypere, men det kan være vanskelig å få til på så kort varsel. Høsten 2022 bør det identifiseres evt. ressursbehov for høsten 2023. Må undersøke for raskt man kan få frigjort ressurser. 

Høst

Flere innholdsprosjekter

% Testing av formater underveis



\section{Prosjekter}
	
	\subsection{Gapsanalyse 2022}

Dette prosjektet er en del av Fase 1 av FTS revisjonen. Målet er å synliggjøre eventuelle gap mellom anbefalinger og nåværende situasjon.

Følgende ressurser vil brukes

\begin{itemize}
	\item FTS prinsippene
	\item FTS kompetanseprofiler
	\item Dybdeevaluering av MTFYMA og BFY 2022
\end{itemize}

Basert på gapsanalysen utvikles det en liste med nødvendige tiltak for å redusere gapet.

Gapsanalysen skal være ferdigstilt innen 2022/12
	\subsection{Visjon for studieprogrammene 2022}

Som et utgangspunkt for videre diskusjon av innholdet i studieprogrammet kan det være hensiktsmessig å få skrevet ned et sett med overordnede mål for studieprogrammene (MTFYMA og BFY). Det kan være nyttig som et grunnlag for å styre videre diskusjoner og arbeid med et mer detaljert innhold i studieprogrammene. Det utvikles også overordnede visjoner for de eksisterende studieretningene.

Sammen med visjonene utformes også en beskrivelse av hva som er forskjeller og likheter mellom programmene.

Prosess:
\begin{itemize}
	\item Kjernegruppen utvikler et utkast til overordnede målsetninger. Dette legges frem på ledergrupper på IMF og IFY før sommeren 2022. 
	\item Tilbakemelding og innspill gis til kjernegruppen frem til 2022/10. 
	\item Kjernegruppen utformer en endelig versjon innen utgangen av 2022.
\end{itemize}




% Langtidsplan

% Forankring

% Hvilket behov ser faggruppene for kunnskap og ferdigheter som er nødvendige for å kunne ta fag som faggruppen er 'ansvarlig' for i høyere årskurs. Hvilke ferdigheter er nødvendige for å ta masteroppgaver? Er det tidligere observert noen generelle mangler?

% Hvordan bør dette dokumentet struktureres?

\section{Ressursbehov}

% Oppsummering av ressursbehov

\chapter{Gapsanalyser}

Dette kapittelet inneholder en sammenlikning av studieprogrammene i forhold til ulike rapporter som har analysert hva som vil være ettertraktet kompetanse hos fremtidige studenter. I tillegg er det gjort et forsøk på et sammendrag av tiltak basert på gapsanalysen i Forslag til endringer.

\section{Forslag til endringer}

Her følger en punktliste over foreslåtte endringer i studieprogrammene baserte på gapsanalysere i forhold til FTS, dybdeevalueringen, oppsamlede erfaringer som er gjort men dagens studieprogram.

Tilsammen vil mengden av de punktene man mener er fornuftige på denne listen diktere om det er nødvendig med en omfattende omlegging eller mindre justeringer.

\begin{itemize}
	\item Innholdet i studieprogrammet beskrives i et Hoveddokument.
	\item IMF og IFY må ha et mye tettere samarbeid om innholdet i emnene slik at man oppnår synergier og ikke bare suboptimalisering ved å ha et felles studieprogram.
	\item Det opprettes egne emner for å utvikle praktiske ferdigheter, spesielt innen eksperimentelt arbeid og beregninger, gjerne kombinert.
	\item Det må være en helhetlig plan for opplæring og trening i numeriske metoder og algoritmer.
	\item Studentene må få opplæring og trening i verktøy som er viktig for effektivt å gjennomføre beregningsorienterte oppgaver. Dette inkluderer: Versjonskontroll (git), package-manager (conda), enhetstesting (pytest), terminalvindu, linux.
	\item Studentene skal få opplæring og trening i å bruke HPC ressurser, parallellisering og optimalisering (kan muligens være valgbart). 
	\item Studentene skal få trening i metoder for maskinlæring. 
	\item Studentene skal få trening i å håndtere store, ustrukturerte datasett.
	\item Eksperimentelle ferdigheter skal struktureres rundt å kunne gjennomføre hele verdikjeden fra problemformulering til rapportering.
	\item Studentene skal få mye mer trening i rapportskriving enn i dag.
	\item 
\end{itemize}

\section{Fremtidens teknologistudium}
	\subsection{FTS prinsippene}

FTS har definert 10 prinsipper\footnote{\url{https://www.ntnu.no/fremtidensteknologistudier/prinsipper}} som NTNU har vedtatt skal være styrende for teknologi-utdanningene ved NTNU. Det gis her en kortfattet vurdering i hvilken grad MTFYMA og BFY oppfyller disse prinsippene. Dersom det eksisterer et gap mellom ønsket og nåværende situasjon gis det kortfattet forslag om hvilke tiltak som kan iverksettes\footnote{Denne gapsanlysen ble ferdigstil xx-yyyy}.

\begin{quote}
	\textbf{Prinsipp 1:} NTNUs teknologistudier skal legge aktivt til rette for at kandidatene, med utgangspunkt i et solid faglig fundament, opparbeider helhetlig og integrert kompetanse, herunder bærekraftkompetanse og digital kompetanse på høyt nivå.
\end{quote}

Studieprogrammet gir studentene at omfattende faglig fundament i fysikk og matematikk, men det er en mangel på trening i å kunne bruke kunnskap og ferdigheter på tvers av flere kunnskapsområder (dvs. integrert kompetanse, e.g. matematisk modellering, multifysikk, eksperimenter, numerikk, prosjektstyring, etc.) og da spesielt i mer omfattende og åpne oppgaver.

Det eksisterer ingen overordnet plan for utvikling av digital kompetanse, noe som medfører at utfallet er usikkert, tilnærmingen fragmentert og nivået for lavt ettersom man ikke vet hva man kan bygge videre på.

Det er uklart hva som menes med \emph{bærekraftskompetanse} for MTFYMA og BFY. Det må defineres hva som ligger i dette for at man skal kunne vurdere om det er oppfylt.

\textbf{Forslag til tiltak:}
\begin{itemize}
	\item Det anbefales at det i større grad utformes læringsaktiviteter hvor studentene må bruke kunnskap og ferdigheter fra flere områder. Dette vil være ganske omfattende tidsmessig så vil kanskje måtte gjøres i dedikerte emner.
	\item Det utarbeides en helhetlig plan over hvordan de digital ferdigheten i programmet utvikles.
	\item Det utarbeids en definisjon av hva man legger i bærekraftskompetanse og dernest en plan for hvordan studentene evt. kan oppnå denne kompetansen.
\end{itemize}

\begin{quote}
	\textbf{Prinsipp 2:} NTNU skal legge aktivt til rette for at kandidater fra teknologistudiene opparbeider tverrfaglig samhandlingskompetanse, og for at man over den samlede studentpopulasjonen får et mangfold i kunnskapsprofiler, samtidig som den enkelte student oppnår tilstrekkelig programfaglig dybde.
\end{quote}

Det er mulig å velge ulike kunnskapsprofiler i de nåværende studieprogrammene men det er i liten grad synliggjort overfor studentene hvordan flere emner kan settes sammen til en hensiktsmessig profil. Data på emnevalg indikerer at studentene ikke velger så bredt som man kunne ønske. Det er også i liten grad synliggjort for studentene hvordan de kan bruke emner fra andre institutt for å skape en helhetlig profil.

Utover EiT er det ingen planlagte elementer i studieprogrammet for å bygge tverrfaglighet. Det er usikkert om EiT i tilstrekkelig stor grad definerer en reell teknologisk problemstilling som et tverrfaglig team skal løse i praksis. 

\textbf{Forslag til tiltak:}

\begin{itemize}
    \item Planlegge og designe tydelige fagprofiler som synliggjøres for studentene.
	\item Lage emner med prosjekter hvor man er nødt til å bruke fagkunnskap fra ulike studieprogram for å komme frem til en løsning.
	\item Samhandlingskompetansen som introduseres i EiT bør nok også introduseres på et tidligere tidspunkt i studiene slik at man både kan nyttiggjøre seg den gjennom studiene og få mer trening i det.
\end{itemize}

\begin{quote}
	\textbf{Prinsipp 3:} Kontekstuell læring skal legges til grunn som gjennomgående pedagogisk prinsipp i NTNUs teknologistudier.
\end{quote}

Kontekstuell læring brukes i svært liten grad. Det er noe uklart hva slags kontekst som vil bidra til bedre læring. Er det primært en anvendt, realistisk kontekst som er viktig?

\textbf{Forslag til tiltak:}

\begin{itemize}
    \item Innhente eksempler fra industri som kan brukes som utgangspunkt for å presentere teori for studentene. Dette kan inkludere gjesteforelesninger og at problemstillingen inkluderes i øvinger og mindre prosjekter.
\end{itemize}

\begin{quote}
	\textbf{Prinsipp 4:} NTNUs teknologistudier skal benytte kunnskapsbaserte, studentaktive og engasjerende undervisnings- og vurderingsformer som er samstemt med utdanningenes overordnede kompetansemål, fremmer god læringskultur, og gir effektiv dybdelæring
\end{quote}

Undervisningsformer ved programmet er i stor grad basert på tradisjon heller enn en kunnskapsbasert tilnærming til hvordan læring fungerer. Ikke dermed sagt at dagens form er feil, men det må vurderes om dagens form bidrar til å oppnå det man ønsker med studieprogrammet.

Emnene er i stor grad preget av stofftrengsel, spesielt på grunnivå, noe som reduserer muligheten for dybdelæring og gir redusert mestringsfølelse, noe som reduserer engasjement og motivasjon.

\textbf{Forslag til tiltak:}

\begin{itemize}
	\item Tydelig plan om innhold i emner på programnivå for å hindre stofftrengsel og forbedre samkjøring.
	\item Forsterke emnegruppene på IFY for å skape mer dialog om undervisningsformer.
	\item Gjør beviste valg av undervisnings- og vurderingsformer slik at det bidrar til god læringskultur.
\end{itemize}

\begin{quote}
	\textbf{Prinsipp VI:} Kvaliteten i NTNUs teknologistudier skal utvikles gjennom en programdrevet tilnærming, i kombinasjon med strategisk porteføljeutvikling og -forvaltning på tvers av programmer og programtyper.
\end{quote}

Studieprogrammene er i svært liten grad programdrevet ettersom det i liten grad eksisterer noen skriftlig dokumentasjon på hva innholdet er. Det lille som eksisterer i beskrivelsen av studieprogrammet er for vag og brukes i liten grad. Emnebeskrivelsen kan fritt endres av faglærer og er ikke styrt av programmet. Faglærerer tenker i liten grad utover sitte eget emne og i liten grad på hvordan det bidrar til helheten når de underviser.

Det er i liten grad tenkt på hvordan emnene henger sammen og bygger videre på hverandre. Eventuelle koblinger forvitrer også som følge av den enkelte faglærers frihet til å endre form og innhold i emnene.

Undervisning sees nok fortsatt på som et individuelt ansvar for et gitt emne og ikke et kollektivt ansvar. Dette er kanskje noe i endring på IFY som følge av etablering av emnegrupper

\textbf{Forslag til tiltak:}

\begin{itemize}
	\item Utvikle dette dokumentet slik at det dokumenterer en felles forståelse av hvordan studieprogrammene bør drives.
	\item ØKt fokus emnegrupper på IFY for å styrke den kollektive tankegangen om undervisning.
\end{itemize}

\begin{quote}
	\textbf{Prinsipp VII:} NTNUs kvalitetsarbeid i teknologistudiene skal stimulere studieprogrammenes utvikling mot utdanningskvalitet i verdensklasse ved å fokusere på kontinuerlig forbedring og systematisk utvikling av kvalitetskultur.
\end{quote}

NTNUs kvalitetssystem har to store mangler. For det første er det primært et rapporteringssystem, men sirkelen er i liten grad sluttet ved at det også lages handlingsplaner som skaper endring. I tillegg er emneevaluering på emnenivå tungt basert på hva studentene \emph{syns}, og i liten grad på hva de faktisk lærer. 

Det er noe benchmarking internasjonalt og interessegrupper i forbindelse med dybdeevalueringer. Dette kunne kanskje styrkes ved å ha fast partnere som man bruker som benchmarking over tid. 

Man bør også kanskje ha blikket mer rettet i alle retninger utover, heller enn å kunne få et blikk på sin egen utdanning en gang hvert femte år.

\begin{quote}
	\textbf{Prinsipp VIII:} NTNU skal gi høy prioritet til strategisk og operativt internasjonalt samarbeid om utvikling av teknologistudier, med mål om å bli et internasjonalt synlig og anerkjent universitet også på dette området.
\end{quote}

Programmene og IFY er liten grad aktive i fora hvor det diskuteres undervisning av ingeniører og realister internasjonalt. IMF har vært mer aktive her. Det bør vurdere om man har kapasitet og om det vil være hensiktsmessig å øke denne aktiviteten.

\begin{quote}
	\textbf{Prinsipp IX:} NTNUs teknologistudier skal vektlegge systematisk samhandling med arbeidsliv og samfunn, med mål om å fremme arbeidsrelevans, legge til rette for livslang læring, og sikre at studenter kan opparbeide relevant arbeidslivserfaring gjennom studiene.
\end{quote}

Studiene har i liten grad samarbeid med eksternt arbeidsliv i studiene.

Et unntak er medisinsk fysikk hvor ansatte på sykhuset er tett integrert i utforming og gjennomføring av undervisning i disse områdenee.

\subsection{Prinsipper primært for andre nivå}

Prinsipp V og prinsipp X er primært relevant for henholdsvis institutt og sentralt nivå. Disse prinsippene er.

\begin{quote}
	\textbf{Prinsipp V:} NTNU skal stille tydelige forventninger til, og gi solid støtte for, kompetanseutvikling hos undervisningspersonell.
\end{quote}

\begin{quote}
	\textbf{Prinsipp X:} NTNU skal utvikle sitt læringsmiljø – og spesielt sin campus og infrastruktur (både fysisk og digital) – i en retning som understøtter de øvrige FTS-prinsippene I – IX, og som fremmer læring, helse og trivsel blant studenter og ansatte.
\end{quote}


	\subsection{fts-competencies}

\subsubsection{K1: Vise fagkunnskap og faglig fundert perspektiv}

Dagens kandidater ved MTFYMA og BFY opparbeider uten tvil en veldig bred faglig kunnskapsbase. Likevel utdypes dette punktet litt mer i detalj og i forhold til noen av disse områdene er det et vist gap mellom ønsket og dagens situasjon. 

Ordlyden og utdypningen presisserer at det som søkes er dyp, virksom kunnskap som søkes, slik at kandidaten er i stand til å \emph{bruke} kunnskapen kreativt og effektivt i problemløsning. Det kan stilles spørsmålstegn ved om dagens eksamensfokus tester kunnskapen relativt overfladisk og over et kort tidsrom, samt at kandidatene får lite erfaring i å anvende kunnskapen. Det vurderes i liten grad om kunnskapen på sikt er dyp og virksom.

Delrapport 1 deler opp dette punktet i 4 deler

\begin{itemize}
	\item ...
\end{itemize}

generaliserbare konsepter.
kontekstualisering
beregningsorientering
stordata, maskinlæring
foretningsforståelse, invoasjonsprosesser

ii) bredde teknisk: prosjektledelse, muliggjørende teknologier

iii) vei dybde vs bredde

iv) komplementært i forhold til fremtidens behov

-bredde i kunnskapsprofiler

Steam - kreativitet

K2 Analyse av komplekse problemstillinger og systemer

K3 Design og implementering av bærekraftige løsninger.


%Dybdeevaluering 2020
\section{Dybdeevaluering 2020}

En ekstern komite og en studentkomite gjennomførte i 2020 en evaluering av studieprogrammene MTFYMA og BFY. Evalueringen fokuserte på \emph{eksperimentelle ferdigheter} og \emph{numeriske ferdigheter}. Hensikten var å dekke beregningsorientering av både fysikk og matematikk men pga av misforståelser i mandatet fokuserte evalueringene primært på fysikkemnene.

Her følger en oppsummering av anbefalinger fra rapporten, tolket av studieprogramledelsen.

\subsection{Generelt}

\begin{itemize}
	\item Studentene bør eksponeres for oppgaver som integrerer beregningsorientering og eksperimenter, samt få trening i å vurdere om spørsmål besvares best med eksperimenter, beregninger eller ny teori eller en kombinasjon av disse.
	\item Studentene bør selvstendig eller i team kunne levere på den praktiske gjennomføringen og må dermed få opplæring i og erfaring med hele verdikjeden fra problemformulering til rapportering.
	\item Studentene bør få erfaring med å jobbe i prosjektform hvor man jobber individuelt i et team og må planlegge hensiktsmessig arbeidsdeling. Studentene bør også få prosjektoppgaver hvor de jobber individuelt.
	\item Studentene bør få en tydelig innføring i normer og regler relatert til sitering, plagiat, IPR m.m.
\end{itemize}

\subsection{Eksperimentelle ferdigheter}

\begin{itemize}
	\item Studentene bør få mer trening i rapportskriving og erfaring med at rapporter kan ha ulike format i ulike situasjoner, ikke nødvendigvis bare som vitenskapelig artikkel. Det bør være felles læringsressurser og man må ha kontroll på at alle studieretningen får tilstrekkelig god trening i rapportering. 
	\item Studentene må få en systematisk innføring i dokumentasjonspraksis, journalføring, dataintegritet m.m. Dette gjelder både eksperimentelt og beregningsorientert arbeid. Det bør refereres til internasjonale standarder om dette. Studentene bør også eksponeres for elektronisk journalføring, ikke kun håndskrevne. Studentene bør kjenne og forstå FAIR prinsippene for vitenskapelig arbeid og generelt om tankesett og teknologi relatert til open science.
	\item Studiet bør prioritere generiske laboratorieferdigheter foran demonstrasjoner for å støtte opp om teori. Laboratoriundervisning kan ha ulike målsetninger og det bør være tydelig for en gitt undervisningsaktivitet hva som er hensikten.
	\item Det bør være mindre bruk av ferdige oppsett av eksperimenter og mer fokus på at studenten må gjennomføre hele verdikjeden fra problemformulering til rapportering. Spesielt å kunne formulere hypoteser som kan testes av eksperimenter eller numeriske beregninger. Dette vil også innebære mer fokus på å kunne redegjøre for antagelser som er gjort i en analyse. Dette er en forutsetning for meningsfylt rapportskriving.
	\item For å kunne designe eksperimenter forutsetter det at studentene kan bruke sentrale måleinstrumenter og har kunnskap om sentral måleprinsipper. Metoder for automatisk datainnsamling (av store data) er spesielt viktig.
	\item Aspekter relatert til eksperimentdesign som studenten må få trening i: vurdere nødvendig nøyaktighet, vurdere ulike oppsett, identifisere og kvantisere feil/usikkerhet, kalibrering, avpasse oppsett i henhold til tid/ressurser, 
	\item Studentene bør ha kjennskap til en bredt spekter av metoder for å analysere data og trening i å anvende disse.
	\item Det bør være egne emner for eksperimentell aktivitet.
	\item Risikoanalyse bør ikke kun inkludere HMS men også prosjektrisiko.
\end{itemize}

\subsection{Beregningsorienterte ferdigheter}

\begin{itemize}
	\item Studentene bør få en systematisk innføring i dokumentasjonspraksis, versjonskontroll og deling av kode.
	\item Studentene bør få trening i distribuerte utviklingsprosjekter hvor flere arbeider med deler av at større prosjekter.
	\item Studenten bør få trening i enhetstesting.
	\item Det bør gis en helhetlig innføring i numeriske beregninger og ferdighetsstrengene anses som et positivt virkemiddel for å nå dette målet.
	\item Studentene bør få kompetanse på maskinlæring og bruk av HPC ressurser.
	\item Studentene bør ha kjennskap til optimalisering og parallellisering
	\item Ulikhet i obligatoriske emner mellom MTFYMA og BFY gjør det vanskelig å bygge på tidligere emner, noe som er en forutsetning for at høyere grads emner er tilstrekkelig avanserte.
	\item Studentene bør kunne håndtere store, uorganiserte data.
	\item Studentene bør få trening i å bruke bereningsmetoder mot multifysikk-problemer.
	\item Studente bør få trening i å bruke både kommersielle og åpne programpakker.
	\item Studenene bør kjenne, forstå og kunne anvende sentral numeriske algoritmer/metoder og forstå deres begrensninger. 
	\item Studentene bør få trening i algoritmisk tekning.
	\item Studenene bør få erfaring med programvare for symbolske beregninger.
	\item Studentene bør ha erfaring med både skriptede og kompilerte språk.
\end{itemize}

\subsection{Studentevaluering}

\begin{itemize}
	\item Det bør gis spesifikk opplæring i ferdigheter som studentene forventes å kunne beherske eller kjenne til. Dette inkluderer:
	\begin{itemize}
		\item Rapportskriving
		\item Journalføring
		\item LaTeX
		\item Feilanalyse
		\item Relevante biblioteker i Python
	\end{itemize}
	\item BFY bør kanskje ha obligatoriske emner med mer selvstendig laboratoriarbeid.
	\item Mer relevant ITGK (ITGK blir lagt om høsten 2022 med mer fokus på numerikk.)
\end{itemize}

% \section{CDIO}
% Gap analysis relative to CDIO standards

% \section {IOP, APS, ...}
% Gap analysis relative to professional societies

% \chapter{Organisasjon}
% Describe the organization that is necessary to maintain this document and ensure its continued use.

% Hvordan sørge for at prosjektet ikke er personavhengig.

% \chapter{Dokumenthåndtering}
% Description of how the document is manged, updated, communicated.
% Plasseres på github fordi ...., github bruker til ... overføres når...

\end{document}
