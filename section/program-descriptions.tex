\section{Overordnet målsetning for programmene}

Felles for MTFYMA og BFY er at de skal gi en solid innføring i matematikk of fysikk og de nødvendige ferdigheter som trengs for å kunne utøve disse fagfeltene.

MTFYMA gir i tillegg til det faglige fundamentet også kunnskap og ferdigheter ikke-faglige ferdigheter som er viktige som ansatt i en bedrift. Studiet består i stor grad obligatoriske for effektivt å kunne oppnå faglig bredde.

BFY tilstrekkelig valgfrihet mot slutten av studiet til å kunne begynne spesialisering mot ulike fagfelt både innen og utenfor fysikk, og sørge for at studenten kan kvalifisere seg til opptak mot disse studiene. Kombinasjonen BFY + MSPHYS gir mulighet for dypere spesialisering i fagområder innen fysikk, mens muligheten for kombinasjonen BFY + annet masterprogram, gir en bredde av kompetanseprofiler.

