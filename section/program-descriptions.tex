\chapter{Overordnede målsetninger}

Dette kapittelet beskriver de overordnede målsetningene med programmet. Hensikten med å beskrive målsetningene på dette nivået er at det gir et oversiktlig sett med målsetninger som kan fungere som rettesnor og gi en felles forståelse av hva vi ønsker å oppnå med programmet\footnote{Dette er bare et utkast. Det planlegges at felles målsetninger utarbeides i løpet av høsten 2022}.

En slik beskrivelse skal ikke være et minste felles multiplum som kan aksepteres enstemmig. Det vil være uenighet (og kontinuerlig diskusjon) om innholdet. De overordnede målsetningene representerer fagmiljøets majoritet, eksterne avtakere og føringer fra ledelse.

Utsagn som ikke virker begrensende for innholdet eller som for de fleste innen fagmiljøet er selvfølgelige, bør unngås.

\section{MTFYMA}

%Brynjulf
\begin{itemize}
	\item MTFYMA skal være et attraktivt studieprogram som tiltrekker seg elever fra videregående skole som har en spesiell interesse for fysikk og matematikk.
	% Hva ligger i å være attraktivt? Må man ha spesielt talent?
	\item MTFYMA skal ha klare kompetansemål for utdanningen som gjør våre kandidater attraktive for framtidige arbeidsgivere.
	\item MTFYMA skal være et integrert studieprogram med en egen tydelig identitet som i størst mulig grad er uavhengig av spesialisering i senere del av studiet.
	\item MTFYMA skal ha en tydelig profil innenfor fagområdene fysikk og matematikk som gjennom hele studieløpet kjennetegnes ved at det har en høyere målsetning innen disse fagene enn øvrige sivilingeniørprogrammer.
	\item MTFYMA skal være et umiskjennelig sivilingeniørprogram.
	\item MTFYMA skal utdanne kandidater som tilfører arbeidslivet kunnskaper om den til enhver tid nyeste metodikken som er tilgjengelig innen fysikk og matematikk for å arbeide med industrielle problemstillinger.
	\item MTFYMA vil etterstrebe et godt læringsmiljø for studentene, gode studieforhold, aktive undervisningsmetoder og ligge i front når det gjelder vurderingsformer og samstemt undervisning.
	\item MTFYMA skal ha et tett og godt forhold til industri og andre avtakere av våre studenter, lytte til deres ønsker og behov og være åpen for å modernisere og tilpasse studieprogrammet etter hva som er etterspurt.
\end{itemize}

%Magnus
\begin{itemize}
	\item MTFYMA gir, I tillegg til det teoretiske grunnlaget, nødvendige ferdigheter i metoder og verktøy som gjør kandidatene til effektive problemløsere.
	\item MTFYMA skal gi studentene trening og trygghet i å anvende sammensatt kunnskap og ferdigheter for å løse komplekse problemstillinger.
	\item MTFYMA er et studieprogram som i stor grad er programmert (obligatoriske emner) for å effektivt oppnå ønsket kunnskap og ferdigheter.
	\item MTFYMA gir i tillegg til det faglige fundamentet også kunnskap og ferdigheter som er viktige som ansatt i en bedrift.
\end{itemize}

\section{BFY}
\begin{itemize}
	\item BFY gir tilstrekkelig valgfrihet mot slutten av studiet til å begynne spesialisering mot ulike fagfelt både innen og utenfor fysikk, og sørge for at studenten kan kvalifisere seg til opptak til disse studiene.
\end{itemize}
%Kombinasjonen BFY + MSPHYS gir mulighet for dypere spesialisering i fagområder innen fysikk, mens muligheten for kombinasjonen BFY + annet masterprogram, gir en bredde av kompetanseprofiler.

