\subsection{fts-competencies}
FTS delrapport 1 introduserer begrepet kompetanseprofil for et studium og beskriver 12 kompetansemål som tilsammen gir en kompetanseprofil som skal kjennetegne studenter fra et teknologistudier ved NTNU.

\subsubsection{K1: Vise fagkunnskap og faglig fundert perspektiv}

Dagens kandidater ved MTFYMA og BFY opparbeider uten tvil en veldig bred faglig kunnskapsbase. Likevel utdypes dette punktet litt mer i detalj og i forhold til noen av disse områdene er det et vist gap mellom ønsket og dagens situasjon. 

Ordlyden og utdypningen presisserer at det som søkes er dyp, virksom kunnskap som søkes, slik at kandidaten er i stand til å \emph{bruke} kunnskapen kreativt og effektivt i problemløsning. Det kan stilles spørsmålstegn ved om dagens eksamensfokus tester kunnskapen relativt overfladisk og over et kort tidsrom, samt at kandidatene får lite erfaring i å anvende kunnskapen. Det vurderes i liten grad om kunnskapen på sikt er dyp og virksom.

Delrapport 1 deler opp dette punktet i 4 deler

\begin{itemize}
	\item ...
\end{itemize}

generaliserbare konsepter.
kontekstualisering
beregningsorientering
stordata, maskinlæring
foretningsforståelse, invoasjonsprosesser

ii) bredde teknisk: prosjektledelse, muliggjørende teknologier

iii) vei dybde vs bredde

iv) komplementært i forhold til fremtidens behov

-bredde i kunnskapsprofiler

Steam - kreativitet

\subsectin{K2: Analyse av komplekse problemstillinger og systemer}

\subsection{K3: Design og implementering av bærekraftige løsninger}

\subsection{K4: Benytte relevante metoder og verktøy}
Det sentrale innholdet i dette punktet er at kandidatene skal beherske metoder og verktøy som gjør dem i stand til å \emph{effektivt anvende} sin fagkunnskap for å finne løsninger på problemer. Rapporten fremhever særskilt noen områder:

\begin{itemize}
	\item Håndtere store, ustrukturerte datasett.
	\item Maskinlæring.
	\item Digital sikkerhet og dømmekraft.
	\item Prosjektplanlegging.
	\item Effektiv kommunikasjon.
	\item Sensorteknologi, automatisering.
\end{itemize}

Generelt for MTFYMA og BFY legges det for stor vekt på teoretisk kunnskap og for lite på verkøty og metoder som er nødvendig for å effektivt bruke denne kunnskapen. Alle punktene i listen over er relevante (kanskje med unntak av digital sikkerhet) og er forbedringsområder. Flere slike metoder er fremhevet i dybdeevalueringen. Det anbefales at programmene, spesielt MTFYMA legger større vekt på opplæring i relevante metoder og verktøy for å anvende kunnskapen og at studentene får tid til å opparbeide erfaring.

\subsection{K5: Drøfte konsekvenser og fremtidsscenarier}
Utvikling av disse ferdighetene krever at studentene eksponeres for oppgaver hvor det er relevant. På lokal skala kan dette innebære å måtte gjøre beviste valg av metodikk for å besvare et problem selv når beslutningsgrunnlaget er usikkert. For å utvikle disse ferdighetene på større nivå (samfunn, økonomi, klima) er det nok nødvendig med realistiske problemstillinger, noe som vil kreve at man har samarbeid med industri.

\subsection{K6: Kjenne til forskning og bidra til teknologiutvikling}
Erfaring med å bidra til forskningsprosjekter. Lede teknologiutvikling.

\subsection{K7: Innhente og kritisk vurdere informasjon}
For studenter ved MTFYMA og BFY anses det som spesielt viktig at de skal være i stand til å vurdere kvaliteten på forskningsresultater og konklusjonene som blir trukket basert på dem. Det virker å være veldig lite innslag av dette i studiet per i dag.

\subsection{K8: Livslang læring}
Dette punktet peker på at studiet ikke bare skal gi studentene kunnskap og ferdigheter men også skal gi dem effektive strategier for læring. Den velkjente frasen fra studenter om at det man lærte mest ved studiet var å lære, er nok en sannhet med modifikasjoner. Mange studenter bruker nok ganske ineffektive metoder for læring. Punktet omhandler også å utvikle en postiv holdning til omstilling og evne til å vurdere sine egne ferdigheter.

Per i dag er det ingen bevist tilnærming ved studiene for hvordan man sørger for at studentene tilegner seg hensiktsmessige læringsstrategier. Det bør nok utvikles.

\subsection{K9: Bruk og refleksjon over normer og helhetstenkning rundt etikk og bærekraft} 
Studieprogrammenes fagfelt er nok såpass lite anvendt at det er vanskelig å se at man kan se på slike problemstillinger uten at det blir syntetisk.

Noe innsikt i etikk er dekket av Ex.Phil. 

Fysmat gir et godt grunnlag for videre studier mot bærekraft. I stedenfor å bake inn i kjernen av studiet kunne man definere en valgfri spesialisering inn mot bærekraft.

\subsection{K10: Målrettethet, samhandlingsevne og lederskap}

Bør få trening i alle disse delene i prosjekter.

\subsection{K11: Kommunikasjon, formidling og dialog}

Bør få trening i alle disse.

\subsection{K12: Nyskaping}

Bør sette opp som en spesialisering.