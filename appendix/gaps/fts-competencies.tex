\subsection{FTS kompetansemål}
FTS delrapport 1\footnote{\url{https://www.ntnu.no/fremtidensteknologistudier}} introduserer begrepet kompetanseprofil for et studium og beskriver 12 kompetansemål som tilsammen gir en kompetanseprofil som skal kjennetegne studenter fra et teknologistudium ved NTNU.

Her følger en analyse av eventuelle gap mellom ønsket kompetanseprofile (i følge FTS) og nåværende studieprogram\footnote{Foreløpig bare et utkast, planlagt ferdigstilt i samarbeid med fagmiljøene i løpet av høsten 2022}.

\subsubsection{K1: Vise fagkunnskap og faglig fundert perspektiv}

Dette utgjør stammen i studieprogrammets innhold og er et omfattende punkt å detaljere. Rapporten foreslår en måte å strukturere innholdet og legger frem noen overordnede aspekter som er viktig. Fagmiljøene og studieprogrammene må spesifisere hva dette betyr i praksis og hvordan det konkret skal implementeres.

Dagens kandidater ved MTFYMA og BFY opparbeider uten tvil en bred faglig kunnskapsbase. Likevel utdypes dette punktet litt mer i detalj og i forhold til noen av disse områdene er det et vist gap mellom ønsket og dagens situasjon.

For eksempel så presiseres det at det som søkes er dyp, virksom kunnskap som søkes, slik at kandidaten er i stand til å \emph{bruke} kunnskapen kreativt og effektivt i problemløsning. Det kan stilles spørsmålstegn ved om dagens eksamensfokus tester kunnskapen relativt overfladisk og over et kort tidsrom, samt at kandidatene får lite erfaring i å \emph{anvende} kunnskapen. Det vurderes i liten grad om kunnskapen på sikt er dyp og virksom.

\begin{comment}
Delrapport 1 deler opp dette punktet i 4 deler

\begin{itemize}
	\item basiskunnskap
	\item breddekunnskap
	\item dybdekunnskap
	\item kompelementær kunnskap
\end{itemize}

generaliserbare konsepter.
kontekstualisering
beregningsorientering
stordata, maskinlæring
foretningsforståelse, invoasjonsprosesser

ii) bredde teknisk: prosjektledelse, muliggjørende teknologier

iii) vei dybde vs bredde

iv) komplementært i forhold til fremtidens behov

-bredde i kunnskapsprofiler

Steam - kreativitet
\end{comment}

\subsubsection{K2: Analyse av komplekse problemstillinger og systemer}

I dette punktet fremheves (for masternivået) at dette skal omfatte \textquote{\emph{selvstendig} problemformulering og analyse} samt at problemene skal være umedgjørlige. Dagens program har muligens et for stort fokus på å løse ferdig oppstilte problemer som har et entydig svar. Man kan også vurdere om det er for lite krav til at studentene er selvstendige aktører. Selv på masteroppgaver møter kandidatene ofte ferdige problemstillinger og ferdige eksperimentelle oppsett som skal brukes.

\subsubsection{K3: Design og implementering av bærekraftige løsninger}

Det er ikke bærekraftige løsninger som er det sentral i dette punktet men evnen til å design og implementere løsninger på problemer; men at disse løsningen skal vurderes i en samfunnsmessig/bærekraftig kontekst.

Utover et prosjekt i Instrumentering ser det ut som studentene får lite trening i dette. Målet må være at studentene får trening i å finne løsninger som representerer det settet med avanserte kunnskaper og ferdigheter de har.

\subsubsection{K4: Benytte relevante metoder og verktøy}
Det sentrale innholdet i dette punktet er at kandidatene skal beherske metoder og verktøy som gjør dem i stand til å \emph{effektivt anvende} sin fagkunnskap for å finne løsninger på problemer. Rapporten fremhever særskilt noen områder:

\begin{itemize}
	\item Håndtere store, ustrukturerte datasett.
	\item Maskinlæring.
	\item Digital sikkerhet og dømmekraft.
	\item Prosjektplanlegging.
	\item Effektiv kommunikasjon.
	\item Sensorteknologi, automatisering.
\end{itemize}

Generelt for MTFYMA og BFY legges det for stor vekt på teoretisk kunnskap og for lite på verktøy og metoder som er nødvendig for å effektivt bruke denne kunnskapen. Alle punktene i listen over er relevante (kanskje med unntak av digital sikkerhet) og har rom for forbedring. Flere slike metoder er også fremhevet i dybdeevalueringen fra 2020. Det anbefales at programmene, spesielt MTFYMA legger større vekt på opplæring i relevante metoder og verktøy for å anvende kunnskapen og at studentene får tid til å opparbeide erfaring.

\subsubsection{K5: Konsekvensanalyse, risikovurdering og scenariotenkning}
Utvikling av disse ferdighetene krever at studentene eksponeres for problemer hvor slike analyser er relevant. På lokal skala kan dette innebære å måtte gjøre beviste valg av metodikk for å besvare et problem selv når beslutningsgrunnlaget er usikkert. For å utvikle disse ferdighetene på større nivå (samfunn, økonomi, klima) er det nok nødvendig med realistiske problemstillinger, noe som vil kreve at man har et samarbeid med industri.

\subsubsection{K6: Kjenne til forskning og bidra til teknologiutvikling}
Å kunne å bidra til forskningsprosjekter samt lede utviklingsprosjekter er trolig sentrale arbeidsoppgaver for kandidater fra MTFYMA og BFY. Det første er til en hvis grad dekket av masteroppgaven. Det andre er i stor grad fraværende. Det bør tydeliggjøres hva som er forventet nivå på disse punktene innenfor dagens studium.

\subsubsection{K7: Innhente og kritisk vurdere informasjon}
For studenter ved MTFYMA og BFY anses det som spesielt viktig at de skal være i stand til å vurdere kvaliteten på forskningsresultater og konklusjonene som blir trukket basert på dem. Det virker å være veldig lite innslag av dette i studiet per i dag.

\subsubsection{K8: Livslang læring}
Dette punktet peker på at studiet ikke bare skal gi studentene kunnskap og ferdigheter men også skal gi dem effektive strategier for læring. Den velkjente frasen fra studenter om at det man lærte mest ved studiet var å lære, er nok en sannhet med modifikasjoner. Mange studenter bruker nok ganske ineffektive metoder for læring. Punktet omhandler også å utvikle en positiv holdning til omstilling og evne til å vurdere sine egne ferdigheter.

Per i dag er det ingen bevist tilnærming ved studiene for hvordan man sørger for at studentene tilegner seg hensiktsmessige læringsstrategier. Det bør nok utvikles.

\subsubsection{K9: Bruk og refleksjon over normer og helhetstenkning rundt etikk og bærekraft} 
I arbeidslivet vil studentene møte problemstillinger som ikke har en ren teknologisk løsning. Det bør vurderes om studentene skal få mer trening i å angripe slike problemstillinger i løpet av studiet.

\subsubsection{K10: Målrettethet, samhandlingsevne og lederskap}

Studentene bør få trening i alle disse ferdighetene gjennom hele studiet.

\subsubsection{K11: Kommunikasjon, formidling og dialog}

Studentene bør få trening i alle disse ferdighetene gjennom hele studiet.

\subsubsection{K12: Nyskaping}

Dette er trolig vanskelig å realisere i de sentral emnene i studiet. Det er også et spørsmål om alle skal ha kompetanse på dette eller om det bør være en mulighet for dem som er spesielt interessert. Dette kan for eksempel løses gjennom en pakke med emner som omfatter teknologiledelse, k/p-emner og i-emner.
