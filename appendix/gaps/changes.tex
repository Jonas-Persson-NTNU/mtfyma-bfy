\section{Forslag til endringer}

Her følger en punktliste over foreslåtte endringer i studieprogrammene baserte på gapsanalysere i forhold til FTS, dybdeevalueringen, oppsamlede erfaringer som er gjort men dagens studieprogram.

Tilsammen vil mengden av de punktene man mener er fornuftige på denne listen diktere om det er nødvendig med en omfattende omlegging eller mindre justeringer.

\begin{itemize}
	\item Innholdet i studieprogrammet beskrives i et Hoveddokument.
	\item IMF og IFY må ha et mye tettere samarbeid om innholdet i emnene slik at man oppnår synergier og ikke bare suboptimalisering ved å ha et felles studieprogram.
	\item Det opprettes egne emner for å utvikle praktiske ferdigheter, spesielt innen eksperimentelt arbeid og beregninger, gjerne kombinert.
	\item Det må være en helhetlig plan for opplæring og trening i numeriske metoder og algoritmer.
	\item Studentene må få opplæring og trening i verktøy som er viktig for effektivt å gjennomføre beregningsorienterte oppgaver. Dette inkluderer: Versjonskontroll (git), package-manager (conda), enhetstesting (pytest), terminalvindu, linux.
	\item Studentene skal få opplæring og trening i å bruke HPC ressurser, parallellisering og optimalisering (kan muligens være valgbart). 
	\item Studentene skal få trening i metoder for maskinlæring. 
	\item Studentene skal få trening i å håndtere store, ustrukturerte datasett.
	\item Eksperimentelle ferdigheter skal struktureres rundt å kunne gjennomføre hele verdikjeden fra problemformulering til rapportering.
	\item Studentene skal få mye mer trening i rapportskriving enn i dag.
	\item 
\end{itemize}