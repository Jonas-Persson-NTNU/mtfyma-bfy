\section{Dybdeevaluering 2020}

En ekstern komite og en studentkomite gjennomførte i 2020 en evaluering av studieprogrammene MTFYMA og BFY. Evalueringen fokuserte på \emph{eksperimentelle ferdigheter} og \emph{numeriske ferdigheter}. Hensikten var å dekke beregningsorientering av både fysikk og matematikk men på grunn av misforståelser av mandatet fokuserte evalueringene primært på fysikkemnene.

Her følger en oppsummering av anbefalinger fra rapporten, tolket av studieprogramledelsen.

\subsection{Generelt}

\begin{itemize}
	\item Studentene bør eksponeres for oppgaver som integrerer beregningsorientering og eksperimenter, samt få trening i å vurdere om spørsmål besvares best med eksperimenter, beregninger eller ny teori eller en kombinasjon av disse.
	\item Studentene bør selvstendig eller i team kunne levere på den praktiske gjennomføringen og må dermed få opplæring i og erfaring med hele verdikjeden fra problemformulering til rapportering.
	\item Studentene bør få erfaring med å jobbe i prosjektform hvor man jobber individuelt i et team og må planlegge hensiktsmessig arbeidsdeling. Studentene bør også få prosjektoppgaver hvor de jobber individuelt.
	\item Studentene bør få en tydelig innføring i normer og regler relatert til sitering, plagiat, IPR m.m.
\end{itemize}

\subsection{Eksperimentelle ferdigheter}

\begin{itemize}
	\item Studentene bør få mer trening i rapportskriving og erfaring med at rapporter kan ha ulike format i ulike situasjoner, ikke nødvendigvis bare som vitenskapelig artikkel. Det bør være felles læringsressurser og man må ha kontroll på at alle studieretningen får tilstrekkelig god trening i rapportering. 
	\item Studentene må få en systematisk innføring i dokumentasjonspraksis, journalføring, dataintegritet m.m. Dette gjelder både eksperimentelt og beregningsorientert arbeid. Det bør refereres til internasjonale standarder om dette. Studentene bør også eksponeres for elektronisk journalføring, ikke kun håndskrevne. Studentene bør kjenne og forstå FAIR prinsippene for vitenskapelig arbeid og generelt om tankesett og teknologi relatert til open science.
	\item Studiet bør prioritere generiske laboratorieferdigheter foran demonstrasjoner for å støtte opp om teori. Laboratoriundervisning kan ha ulike målsetninger og det bør være tydelig for en gitt undervisningsaktivitet hva som er hensikten.
	\item Det bør være mindre bruk av ferdige oppsett av eksperimenter og mer fokus på at studenten må gjennomføre hele verdikjeden fra problemformulering til rapportering. Spesielt å kunne formulere hypoteser som kan testes av eksperimenter eller numeriske beregninger. Dette vil også innebære mer fokus på å kunne redegjøre for antagelser som er gjort i en analyse. Dette er en forutsetning for meningsfylt rapportskriving.
	\item For å kunne designe eksperimenter forutsetter det at studentene kan bruke sentrale måleinstrumenter og har kunnskap om sentral måleprinsipper. Metoder for automatisk datainnsamling (av store data) er spesielt viktig.
	\item Aspekter relatert til eksperimentdesign som studenten må få trening i: vurdere nødvendig nøyaktighet, vurdere ulike oppsett, identifisere og kvantisere feil/usikkerhet, kalibrering, avpasse oppsett i henhold til tid/ressurser, 
	\item Studentene bør ha kjennskap til en bredt spekter av metoder for å analysere data og trening i å anvende disse.
	\item Det bør være egne emner for eksperimentell aktivitet.
	\item Risikoanalyse bør ikke kun inkludere HMS men også prosjektrisiko.
\end{itemize}

\subsection{Beregningsorienterte ferdigheter}

\begin{itemize}
	\item Studentene bør få en systematisk innføring i dokumentasjonspraksis, versjonskontroll og deling av kode.
	\item Studentene bør få trening i distribuerte utviklingsprosjekter hvor flere arbeider med deler av et større prosjekt.
	\item Studenten bør få trening i enhetstesting.
	\item Det bør gis en helhetlig innføring i numeriske beregninger og ferdighetsstrengene anses som et positivt virkemiddel for å nå dette målet.
	\item Studentene bør få kompetanse på maskinlæring og bruk av HPC ressurser.
	\item Studentene bør ha kjennskap til optimalisering og parallellisering
	\item Ulikhet i obligatoriske emner mellom MTFYMA og BFY gjør det vanskelig å bygge på tidligere emner, noe som er en forutsetning for at høyere grads emner er tilstrekkelig avanserte.
	\item Studentene bør kunne håndtere store, uorganiserte data.
	\item Studentene bør få trening i å bruke bereningsmetoder mot multifysikk-problemer.
	\item Studente bør få trening i å bruke både kommersielle og åpne programpakker.
	\item Studenene bør kjenne, forstå og kunne anvende sentral numeriske algoritmer/metoder og forstå deres begrensninger. 
	\item Studentene bør få trening i algoritmisk tekning.
	\item Studenene bør få erfaring med programvare for symbolske beregninger.
	\item Studentene bør ha erfaring med både skriptede og kompilerte språk.
\end{itemize}

\subsection{Studentevaluering}

\begin{itemize}
	\item Det bør gis spesifikk opplæring i ferdigheter som studentene forventes å kunne beherske eller kjenne til. Dette inkluderer:
	\begin{itemize}
		\item Rapportskriving
		\item Journalføring
		\item LaTeX
		\item Feilanalyse
		\item Relevante biblioteker i Python
	\end{itemize}
	\item BFY bør kanskje ha obligatoriske emner med mer selvstendig laboratoriarbeid.
	\item Mer relevant ITGK (ITGK blir lagt om høsten 2022 med mer fokus på numerikk.)
\end{itemize}