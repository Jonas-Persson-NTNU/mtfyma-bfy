Her følger en prosjektplan for revisjon av programmene.

2022 

06-07 Utvikle gapsanalyser relativt til rapporter/anbefalinger: FTS (prinsipper og kompetanseprofiler), dybdeevaluering 2020. Dette skal resultere i en liste med konkrete anbefalinger om endring av innhold i studieprogrammet.  

07 Presentasjon for fagmiljøene

Høst Innhold- og strukturprosjekter: Overordnet struktur, beregning, eksperimentelt, læringsmiljø, vurderingsformer, undervisningsformer, mekanikk, elmag, matematikk, ønsker faggrupper.

Revider prosjektplan. Større/mindre aktiviteter

Ressursbehov er 20\% frikjøp i prosjekgruppen (50\% i tillegg til programledere) De andre komiteene leverer arbeider som ikke er for arbeidskrevende. 

2023 

Vår Overordnet struktur legges litt fastere, undervisningsformer.

Ressursbehov. som over. Det er relativt usikkert hva ressursbehovet her er, ettersom dette er en relativt ny arbeidsmetode.

% Det kan være behov her for en gruppe som kan gå litt dypere, men det kan være vanskelig å få til på så kort varsel. Høsten 2022 bør det identifiseres evt. ressursbehov for høsten 2023. Må undersøke for raskt man kan få frigjort ressurser. 

Høst

Flere innholdsprosjekter

% Testing av formater underveis

